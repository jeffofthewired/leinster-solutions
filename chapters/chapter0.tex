\section*{Chapter 0}

\begin{enumerate}
    \item[\textbf{0.10}]
    \textcolor{question}{
        Let $S$ be a set.
        The indiscrete topological space $I(S)$ is the space whose set of points
        is $S$ and whose only open subsets are $\emptyset$ and $S$ itself.
        Find a universal property satisfied by the space $I(S)$.
    }

    Define $i:I(S)\to S$ as the identity map.\\
    The universal property for $(I(S), i)$ is as follows:

    For all topological spaces $X$, for all set maps $f:X\to S$,\\
    there exists a unique continuous function $\tilde f: X\to I(S)$
    such that $f = i\circ \tilde f$.

    \begin{center}
        \begin{tikzcd}[column sep=large]
            \forall X \rar[dotted, "\exists !\tilde f"description]\drar["\forall f"'] & I(S)\dar{i}\\
            & S
        \end{tikzcd}
    \end{center}

    \textcolor{question}{
        \textbf{Extension:} Prove that $(I(S), i)$ satisfies this universal property.
    }

    For existence, let $\tilde f = f$, as $f = i\circ f$. Every map to $I(S)$ is continuous, thus $\tilde f$ is continuous.
    For uniqueness, supposing $\tilde g = f$ also satisfies $f = i\circ g = i\circ f$, then $g = f$
    since the identity map $i$ is bijective.

    \textcolor{question}{
        \textbf{Extension:} Prove that any other object satisfying the property is isomorphic to $I(S)$.
    }

    Suppose that $(T,j)$ is another pair which satisfies the universal property.

    Since $(T,j)$ satisfies the universal property, for all $X$ and $f:X\to S$,\\
    namely $I(S)$ and $i:I(S)\to S$, there exists a unique continuous $h:I(S)\to T$\\
    such that $i=j\circ h$.

    
    \begin{center}
        \begin{tikzcd}
            I(S)\rar{h}\drar[leftrightarrow, "i"'] & 
            T\dar{j}\\
            &
            S
        \end{tikzcd}
    \end{center}

    Since $(I(S),i)$ satisfies the universal property, for all $X$ and $f:X\to S$,\\
    namely $I(S)$ and $i:I(S)\to S$, there exists a unique continuous $\tilde f:I(S)\to I(S)$\\
    such that $i = i\circ \tilde f$.

    Note that $i\circ j\circ h: I(S)\to I(S)$ is continuous since the composition of
    continuous maps is also continuous. Since $i$ is the identity map,
    this can be shortened to $j\circ h$.
    We also know that setting $\tilde f=j\circ h$ satisfies $i=i\circ \tilde f$.
    The identity map on $I(S)$ also satisfies this.\\
    By uniqueness, $j\circ h$ is the identity map.

    Similarly $h\circ (i\circ j):T\to T$ is continuous, which this shortens to $h\circ j$.

    Since $(T,j)$ satisfies the universal property, for all $X$ and $f:X\to S$,\\
    namely $T$ and $j:T\to S$, there exists a unique continuous $\tilde f:T\to T$\\
    such that $j = j\circ \tilde f$.

    Note that $j\circ (h\circ j) = (j\circ h)\circ j = i\circ j = j$, thus $\tilde f = h\circ j$
    satisfies this property. But the identity map on $T$ also satisfies it.
    By uniqueness $h\circ j$ is the identity map.\\
    Therefore $T\cong I(S)$.
    

    % \begin{tikzcd}
    %     T\rar{j}\drar["j"'] & 
    %     I(S)\dar[leftrightarrow]{i}\rar{h} & 
    %     T\\
    %     &
    %     S\urar[leftarrow, "j"']
    % \end{tikzcd}


    \pagebreak

    \item[\textbf{0.11}]
    \textcolor{question}{
        Fix a group homomorphism $\phi:G\to H$. Let $\varepsilon$ denote any trivial homomorphism.\\
        Find a universal property satisfied by the pair $(\ker\theta,\iota)$
        of the diagram.
    }

    \begin{center}
        {\color{question}
            \begin{tikzcd}
    \ker(\theta)\arrow[r, hook, "\iota"] & G
    \arrow[r, shift left, "\theta"]\arrow[r, shift right, "\varepsilon"'] & H
\end{tikzcd}
        }
    \end{center}

    The above property can be expressed in terms of the commutative diagram
    \begin{center}
        \begin{tikzcd}

            \ker \theta \rar[hook]{\iota}\arrow[dr, "\varepsilon"'] &
            G\dar{\theta}\\
            & H
        \end{tikzcd}
    \end{center}


    The universal property for $(\ker \theta, \iota)$ is as follows:\\
    For all $K$ and $\phi:K\to G$ such that $\theta\circ\phi = \varepsilon$,\\
    there exists a unique $f:K\to\ker\theta$ such that $\phi = f\circ\iota$.\\

    \begin{center}
        \begin{tikzcd}[row sep=small]
            \forall K\arrow[dd,dotted,"\exists !f"description]\arrow[dr,"\forall\phi"]\arrow[drr, bend left,"\forall\varepsilon"]
            & & \\ &
            G\rar{\theta} & H\\
            \ker\theta\arrow[ur, hook, "\iota"']\arrow[urr, bend right, "\varepsilon"'] & &
        \end{tikzcd}
    \end{center}

    \textcolor{question}{
        \textbf{Extension:} Prove that $(\ker\theta, \iota)$ satisfies this universal property.
    }

    \textbf{Existence}

    Let $K$ be a group and $\phi:K\to G$ be a homomorphism such that $\theta\circ\phi = \varepsilon$.
    By the definition of the trivial homomorphism $\varepsilon$,
    for any $x\in K$, we have $(\theta\circ\phi)(x) = 0$, and thus $\phi(x)\in\ker\theta$.
    Thus the range of $\phi$ is $\ker \theta$. Let $f=x\mapsto \phi(x)$, then $f\circ i = \phi\circ i = \phi$ as required.

    \textbf{Uniqueness}

    Suppose $f':K\to\ker\theta$ such that $\iota\circ f = \phi = \iota\circ f'$.
    The inclusion map $\iota$ is injective so
    \begin{align*}
        \iota\circ f = \iota\circ f' \implies f = f'.
    \end{align*}


    \textcolor{question}{
        \textbf{Extension:} Prove that any other object satisfying the property is isomorphic to $\ker\theta$.
    }

    Let $(Y,\psi)$ be another pair satisfying the universal property.

    Since $(Y,\psi)$ satisfies the universal property, for all $K$ and $\phi:K\to G$,\\
    namely $\ker\theta$ and $\iota$, such that there exists a unique $g$ such that $\psi\circ g=\iota$.

    Since $(\ker\theta,\iota)$ satisfies the universal property, for all $K$ and $\phi:K\to G$,\\
    namely $Y$ and $\psi$, such that there exists a unique $f$ such that $\iota\circ f=\psi$.

    Substituting $\psi\circ g=\iota$ into $\iota\circ f=\psi$ we get $\psi\circ g\circ f = \psi$.\\
    Substituting $\iota\circ f=\psi$ into $\psi\circ g=\iota$ we get $\iota\circ f\circ g = \iota$.

    Since $(Y,\psi)$ satisfies the universal property, for all $K$ and $\phi:K\to G$,\\
    namely $Y$ and $\psi$, such that there exists a unique $h$ such that $\psi\circ h=\psi$.\\
    But both $g\circ f$ and the $\text{id}_Y$ satisfy this property, thus $g\circ f = \text{id}_Y$.\\
    Similarly we can show $f\circ g = \text{id}_{\ker\theta}$, thus proving $Y\cong\ker\theta$.

    \pagebreak

    \item[\textbf{0.12}]
    \textcolor{question}{
        Verify the universal property for the topological space covered by two open subsets \\
        $X = U\cup V$, and the two inclusion maps
        $i':V\to X$ and $j':U\to X$:\\
        For all $Y$ and $g:V\to Y$ and $f:U\to Y$,\\
        there exists a unique $h:X\to Y$ such that $g=h\circ i'$ and $f=h\circ j'$.
    }

    \begin{center}
        {\color{question}
            \begin{tikzcd}
    U\cap V \rar[hook]{i}\dar[hook]{j} & 
    U \dar[hook]{j'}\arrow[ddr, bend left, "\forall f"] &\\
    V \arrow[r, hook, "i'"]\arrow[drr, bend right, "\forall g"'] & 
    X \arrow[dr, dotted, "\exists ! h" description] &\\
    & & \forall Y
\end{tikzcd}
        }
    \end{center}

    I do not know topology so I'll come back to this later.

    \item[\textbf{0.13}]
    \textcolor{question}{
        Denote by $\mathbb Z[x]$ the polynomial ring over $\mathbb Z$ in one variable.
    }
    \begin{enumerate}
        \item 
        \textcolor{question}{
            Prove that for all rings $R$ and all $r\in R$, there exists
            a unique ring homomorphism $\phi_r : \mathbb Z[x]\to R$ such that $\phi_r(x)=r$.
        }

        \textbf{Existence}

        Define $\phi_r$ to map polynomials to their evaluation at $x=r$, that is $\phi_r = p\mapsto p(r)$.\\
        Then $\phi_r(x)=r$ trivially follows. To show that this is a ring homomorphism, we note 
        $$\phi_r(p(x)q(x))=(pq)(r) = p(r)q(r) = \phi_r(p(x))\phi_r(q(x)),$$
        $$\phi_r(p(x)+q(x))=(p+q)(r) = p(r)+q(r) = \phi_r(p(x))+\phi_r(q(x)).$$

        \textbf{Uniqueness}

        Suppose that $\psi_r:\mathbb Z[x]\to R$ is another ring homomorphism such that $\psi_r(x)=r$.\\
        For all $k\in\mathbb N$,
        $$\psi_r(x^k) = \psi_r(x)^k= r^k = \phi_r(x^k).$$
        For each polynomial $p(x)=\sum_{k=0}^n a_kx^k$,
        $$\psi_r\!(p(x)) 
        = \psi_r\!\left(\sum_{k=0}^na_kx^k\right) 
        = \sum_{k=0}^n a_k\psi_r\!\left(x^k\right) 
        =\sum_{k=0}^n a_kr^k
        = \phi_r(p(x)).$$

        \item 
            \textcolor{question}{Let $A$ be a ring and $a\in A$. Suppose that for all rings $R$
            and all $r\in R$, there exists a unique ring homomorphism
            $\phi:A\to R$ such that $\phi(a)=r$. 
            Prove that there is a unique isomorphism $\iota : \mathbb Z[x]\to A$
            such that $\iota (x)=a$.
        }

    \end{enumerate}
    
    \item[\textbf{0.14}]
    \textcolor{question}{
        Let $X$ and $Y$ be vector spaces. For the purposes of this exercise only, a \emph{cone}
        is a triple $(V,f_1, f_2)$ consisting of a vector space $V$, a linear map $f_1:V\to X$
        and a linear map $f_2:V\to Y$.
    }
    \begin{enumerate}
        \item
        \textcolor{question}{
            Find a cone $(P,p_1,p_2)$ with the following property:\\
            For all cones $(V,f_1, f_2)$, \\
            there exists a unique linear map $f:V\to P$,
            such that $p_1\circ f=f_1$ and $p_2\circ f = f_2$.
        }
    \end{enumerate}
\end{enumerate}