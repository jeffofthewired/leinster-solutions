\section*{Chapter 0}

\begin{enumerate}
    \item[\textbf{0.10}]
    \textcolor{question}{
        Let $S$ be a set.
        The indiscrete topological space $I(S)$ is the space whose set of points
        is $S$ and whose only open subsets are $\emptyset$ and $S$ itself.
        Find a universal property satisfied by the space $I(S)$.
    }

    Define $i:S\to I(S)$ as the identity function.\\
    The universal property is as follows:

    For all topological spaces $X$, for all functions $f:X\to S$,\\
    there exists a unique continuous function $\tilde f: X\to I(S)$
    such that $\tilde f = i\circ f$.
    \item[\textbf{0.11}]
    \textcolor{question}{
        Fix a group homomorphism $\phi:G\to H$.\\
        Find a universal property satisfied by the pair $(\ker(\theta),\iota)$
        of the diagram.
    }

    \begin{center}
        {\color{question}
            \begin{tikzcd}
    \ker(\theta)\arrow[r, hook, "\iota"] & G
    \arrow[r, shift left, "\theta"]\arrow[r, shift right, "\varepsilon"'] & H
\end{tikzcd}
        }
    \end{center}

    For all $K$ and $\phi:K\to G$ such that $\theta\circ\phi = \epsilon\circ\phi$,\\
    there exists a unique $f:K\to\ker\theta$ such that $\phi = f\circ\iota$.

    \item[\textbf{0.12}]
    \textcolor{question}{
        Verify the universal property for the topological space covered by two open subsets \\
        $X = U\cup V$, and the two inclusion maps
        $i':V\to X$ and $j':U\to X$:\\
        For all $Y$, and $g:V\to Y$ and $f:U\to Y$,\\
        there exists a unique $h:X\to Y$ such that $g=h\circ i'$ and $f=h\circ j'$.
    }

    \begin{center}
        {\color{question}
            \begin{tikzcd}
    U\cap V \rar[hook]{i}\dar[hook]{j} & 
    U \dar[hook]{j'}\arrow[ddr, bend left, "\forall f"] &\\
    V \arrow[r, hook, "i'"]\arrow[drr, bend right, "\forall g"'] & 
    X \arrow[dr, dotted, "\exists ! h" description] &\\
    & & \forall Y
\end{tikzcd}
        }
    \end{center}

    \pagebreak

    \item[\textbf{0.13}]
    \textcolor{question}{
        Denote by $\mathbb Z[x]$ the polynomial ring over $\mathbb Z$ in one variable.
    }
    \begin{enumerate}
        \item 
        \textcolor{question}{
            Prove that for all rings $R$ and all $r\in R$, there exists
            a unique ring homomorphism $\phi_r : \mathbb Z[x]\to R$ such that $\phi_r(x)=r$.
        }

        \textbf{Existence}

        Define $\phi_r$ to map polynomials to their evaluation at $x=r$, that is $\phi_r = p\mapsto p(r)$.\\
        Then $\phi_r(x)=r$ trivially follows. To show that this is a ring homomorphism, we note 
        $$\phi_r(p(x)q(x))=(pq)(r) = p(r)q(r) = \phi_r(p(x))\phi_r(q(x)),$$
        $$\phi_r(p(x)+q(x))=(p+q)(r) = p(r)+q(r) = \phi_r(p(x))+\phi_r(q(x)).$$

        \textbf{Uniqueness}

        Suppose that $\psi_r:\mathbb Z[x]\to R$ is another ring homomorphism such that $\psi_r(x)=r$.\\
        For all $k\in\mathbb N$,
        $$\psi_r(x^k) = \psi_r(x)^k= r^k = \phi_r(x^k).$$
        For each polynomial $p(x)=\sum_{k=0}^n a_kx^k$,
        $$\psi_r\!(p(x)) 
        = \psi_r\!\left(\sum_{k=0}^na_kx^k\right) 
        = \sum_{k=0}^n a_k\psi_r\!\left(x^k\right) 
        =\sum_{k=0}^n a_kr^k
        = \phi_r(p(x)).$$

        \pagebreak

        \item 
            \textcolor{question}{Let $A$ be a ring and $a\in A$. Suppose that for all rings $R$
            and all $r\in R$, there exists a unique ring homomorphism
            $\phi:A\to R$ such that $\phi(a)=r$. 
            Prove that there is a unique isomorphism $\iota : \mathbb Z[x]\to A$
            such that $\iota (x)=a$.
        }

    \end{enumerate}
    

\end{enumerate}